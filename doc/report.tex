\documentclass[a4paper,11pt, oneside]{book}
\usepackage[utf8]{inputenc}
\usepackage[francais]{babel}
\usepackage[T1]{fontenc}
\usepackage{graphicx}
\usepackage{float}
\usepackage{wrapfig}
\usepackage{setspace}
\usepackage{geometry}
\usepackage{hyperref}
\usepackage{multicol}
\usepackage{etoolbox}
\usepackage{color}
\usepackage[explicit,pagestyles]{titlesec}
\usepackage[absolute,overlay]{textpos}
\usepackage{fancyhdr}
\usepackage[export]{adjustbox}
\usepackage{eurosym}
\usepackage{titlesec}
\usepackage[]{algorithm2e}


% ====== CONFIG ========

%\setmainfont{Roboto Light}

%\setmainfont{Roboto Light}
%\setsansfont{Roboto}
%\setmonofont{Roboto}
%\newfontfamily\light{Roboto Slab Light}
\graphicspath{{img/}}
\setlength{\unitlength}{1mm}
\makeatletter

\definecolor{primary}{RGB}{44, 62, 80}


\titleformat{\chapter}[display]{\huge}{\thechapter \quad #1}{0pt}{}
\titlespacing{\chapter}{0pt}{0pt}{0pt}

\titleformat{\section}[display]{\LARGE}{}{0pt}{\thesection \quad #1}


\setlength{\TPHorizModule}{1mm}
\setlength{\TPVertModule}{1mm}
\def\sizeMedia{38}
\def\size{3.8cm}
\def\sizeMargin{0.2cm}
\def\margin{2}
\def\fixMargin{0}

\pagestyle{plain}

\author{Romain Cahu, Victor Cholley-Barroyer, Yann Prono, Quentin Tardivon}
\date{\today}

\def\school{TELECOM Nancy}
\def\schoolAddress{193 Avenue Paul Muller}
\def\schoolPostalCode{54602}
\def\schoolCity{Villers-lès-Nancy}
\def\schoolCodeAndCity{\schoolPostalCode, \schoolCity}
\def\schoolYear{2016 - 2017}

\def\appName{MALG Project}
\def\widthImage{1}
\def\todo{{\color{red}\Huge{TODO}}}

\def\schoolYear{2016 - 2017}

% ====== END CONFIG ========


\begin{document}

	\begin{titlepage}
		\thispagestyle{empty}

{\color{primary}


\includegraphics[width=4.0cm]{img/school-logo.eps}
\hspace{9mm}
\includegraphics[width=4.0cm]{img/collegium-logo.eps}
\hspace{5mm}
\includegraphics[width=4.0cm]{img/university-logo.eps}

\vspace{0.5cm}

	\begin{center}


			{\color[rgb]{0.8,0.8,.8}\rule{\textwidth}{0.8pt}}
			\vspace{0.2cm}

			\baselineskip=3pt
			{\huge \bfseries{Modèles et algorithmes}}\\
			\vspace{0.5cm}
			{\Large \bfseries{Tri à bulles - Algorithme P1 - Triangle de Floyd}}
			\vspace{0.5cm}

		{\color[rgb]{0.8,0.8,.8}\rule{\textwidth}{0.8pt}}

		\vspace{1cm}

		\Large{Groupe 10}\\

		\vspace{1cm}
		\includegraphics[width=0.4\textwidth]{logo.png}


		\vspace{2cm}

		\Large{Romain Cahu}\\
		\Large{Victor Cholley-Barroyer}\\
		\Large{Quentin Tardivon}\\
		\Large{Yann Prono}\\
		\vspace{1cm}
		\large{\schoolYear}
	\end{center}

}

	\end{titlepage}

	\newpage\newpage\null\thispagestyle{empty}
	\newpage
		\tableofcontents
		\thispagestyle{empty}

	\chapter{Introduction}
	\setcounter{page}{1}

		L'objectif de ce projet est d'analyser des programmes écrits en C et de déterminer pour chacun d'entre eux des
		propriétés validées. Pour cela, on vérifiera l'absence d'erreurs à l'exécution du programme et on montrera la correction partielle des programmes.
		On dispose de la plateforme Toolbox TLA ainsi que de l'outil Frama-C.\\

		\noindent Nous ferons l'analyse des trois programmes C suivants:
		\begin{itemize}
			\item Bubble sort
			\item Floyd triangle
			\item Last algorithm
		\end{itemize}

	\chapter{Triangle de Floyd}

		\section{Description de l'algorithme}

		Le Triangle de Floyd prend en paramètre un entier n et imprime à l'écran
		n lignes contenant chacune 1 puis 2 puis 3 jusque n entier.


		\section{Implémentation et preuves PlusCal - TLA+}

			L'algorithme PlusCal correspondant est relativement simple. On a étiqueté
			un maximum d'étape de l'algorithme pour pouvoir y référer plus facilement
			et pour prouver un maximum de préconditions et d'invariants.

			On a d'abord vérifié les invariants de boucles. En effet, le fait d'être
			dans la boucle p2 (c'est-à-dire que le programme est au point p5) implique
			que i est plus petit que n. De la même façon, en p4 c est plus petit ou égal
			à n.

			On s'est ensuite assuré qu'il n'y avait pas d'overflow ou d'underflow au sein
			du programme. On a pour cela défini le max à 65535 correspondant à l'entier
			maximum représentable dans un système 16bits non signé.

			Enfin, la correction partielle du programme est assurée par la propriété
			Correctness. Lorsqu'on arrive en pc="Done", c'est-à-dire à la fin du programme
			on s'assure que le dernier entier imprimé correspond à la formule suivante:
			$a = [n \times (n + 1) / 2] + 1$

			Cette formule correspond à la somme suivante: $$\sum_{k=1}^{n}k$$

		\section{Implémentation et preuves Frama-C}
		Les annotations ajoutées dans le programme vont permettre de
		vérifier le bon comportement attendu par la fonction main (préconditions et postconditions) mais
		aussi de prouver les assertions tout au long de l'algorithme.\\

		\noindent Il est attendu que la fonction \textit{main()} renvoie 0. Ensuite,
		on s'assure que l'entier n donné par l'utilisateur est un entier n'est pas en overflow ni en underflow.
		Or ici, la variable n représente le nombre de lignes. Si on s'en réfère à la
		formule précédente, on obtiendrait comme valeur finale:
		$$[65535 \times (65535 + 1) / 2] + 1 = 2147450881$$

		\noindent On obtiendrait alors une valeur d'overflow dans ce cas. Afin d'assurer le bon
		fonctionnement de ce programme, il est nécessaire d'ajouter des bornes sur la valeur n lors
		du \textit{scanf()} afin d'éviter l'overflow de telle manière que:
		$$[n \times (n + 1) / 2] + 1 	\leq 65535$$
		\noindent D'autres vérifications sont effectuées sur les boucles afin de prouver la terminaison de ces dernières.

	\chapter{Tri à bulles}

		\section{Description de l'algorithme}

		Le tri bulle consiste à comparer répétitivement les éléments consécutifs d'un tableau,
		et à les permuter lorsqu'ils sont mal triés. Ici, il prend en paramètre un
		tableau d'entiers à trier et le nombre d'éléments de ce tableau.


		\section{Implémentation et preuves PlusCal - TLA+}

		De la même façon que le Triangle de Floyd l'implémentation en PlusCal est
		relativement rapide. La seule difficulté est la déclaration de tableau.

		Ici on a fait la preuve de l'algorithme sur un tableau fixe de 4 éléments.

		De la même façon que le programme précédent on a d'abord vérifier les invariants
		de boucles dans les invariants Prec1 et Prec2.

		On a ensuite vérifié qu'après une opération de swap, les deux éléments
		du tableau étaient localement ordonnés dans la propriété Prec3. On peut également
		vérifier l'absence d'overflow et d'underflow de la même façon que pour le programme
		précédent.

		Enfin, on a vérifié la correction de l'algorithme en s'assurant que le tableau
		résultat était globalement trié. Pour cela on a déclaré une fonction isSorted vérifiant
		que pour tout élément du tableau on tableau[k] <= tableau[k+1].
		On a ensuite appliqué cette fonction à la fin de l'algorithme, c'est-à-dire
		lorsque pc = "Done".


		\section{Implémentation et preuves Frama-C}

		De la même manière que le programme du triangle de Floyd, une vérification
		est faite afin d'éviter l'overflow ou l'underflow sur les entiers
		donnés par l'utilisateur. Des annotations sont également ajoutées
		afin de vérifier la terminaison des deux boucles principales du programme.

		\noindent La principale preuve ici est au niveau du \textit{swap}. On s'assure
		qu'en cas de swap, le swap s'effectue et est correcte.
		À la fin du programme, on vérifie que le tableau est correctement trié.


	\chapter{Algorithme P1}

		\section{Description de l'algorithme}

		L'algorithme P1 était un algorithme inconnu. On s'est rendu compte lors de la
		preuve de ce dernier qu'il calcule les termes de la suite récursive de 2nd
		ordre suivante: $b_n = 2 \times b_{n-1} + 2 \times b_{n-2}$ avec $b_0 = 1$ et $b_1 = 1$.

		\section{Implémentation et preuves PlusCal - TLA+}

		De la même façon que les deux algorithmes précédents, l'implémentation PlusCal
		était relativement simple.

		Toujours comme les algorithmes précédents, on a vérifié les conditions de la boucle
		p9 ainsi que les conditions des tests p5 et p7 respectivement dans les propriétés
		Prec5, Prec1 et Prec3. On a également vérifié le non respect de ces conditions
		dans les blocs else de ces différentes conditions dans Prec2 et Prec4. On a
		également la correction partielle pour v = 0 et v = 1 dans la propriété Res,
		ce qui correspond aux termes $b_0$ et $b_1$ de la suite. On pouvait vérifier
		que les éléments de la suite étaient bien des entiers naturels dans la propriété
		PartialCorrectness. On a également, comme pour les algorithmes précédents vérifié
		l'absence d'overflow et d'underflow.

		En revanche, la correction totale de l'algorithme pose problème. En effet,
		on a effectué la résolution de l'équation caractéristique de la suite pour trouver
		l'expression de son terme général en fonction de $b_0$ et $b_1$.
		On trouve après simplification:
		$$1/2 \times [(1 + \sqrt 3)^n + (1 - \sqrt 3)^n]$$

		Or, TLA+ ne nous permet de réaliser de racine carré ou de puissance avec
		un réel ($1/2$ par exemple).

		On peut en revanche imaginer une version de l'algorithme qui calculerait
		directement le terme n de la suite sans utilisation de boucle for, ce qui
		augmenterait la vitesse de l'algorithme pour des grandes valeurs de n.


		\section{Implémentation et preuves Frama-C}

		Le programme \textit{Last algorithm} est particulier. Son résultat dépend de la valeur du paramètre x.
		Dans le cas où x vaut 0 ou 1, le résultat est direct. C'est pourquoi, on a défini à la signature
		de la méthode différents \textit{behaviors} suivant la valeur de x. Dans le cas où ne vaut
		ni 0 ni 1, le résultat obtenu doit correspondre à la formule ci-dessus. Ayant quelques
		difficultés à utiliser les fonctions mathématiques qui sont d'après la documentation de Frama-C,
		des \textit{built-in}. Nous n'avons pas pu définir correctement le comportement de la méthode dans ce cas-là.
		On s'assure enfin que la boucle \textit{while} termine.

	\chapter{Conclusion}

		Ce projet nous a permis d'appliquer les notions vues en cours de MALG sur les
		preuves d'algorithmes à l'aide des outils TLA+/PlusCal et Frama-C. On a ainsi
		pu démontrer un certain nombre de propriétés, dont la correction partielle et
		des propriétés de suretés, sur trois algorithmes: le triangle de Floyd,
		le tri à bulles et un algorithme "inconnu". Ce projet donne une vue d'ensemble
		des possibilités de ces outils mais aussi de leurs limitations sur des
		algorithmes classiques et sur certains algorithmes particuliers, autant en
		termes de calcul que de structures.

\end{document}
